\documentclass[UTF8]{ctexart}
\usepackage[svgnames]{xcolor}
\usepackage{geometry, hyperref, fancyhdr, enumitem, lastpage, changepage}
\geometry{a4paper, hmargin = 2.53cm, vmargin = 2.53cm}
\ctexset{
	section/format = {\large\bfseries},
	section/beforeskip = {3.9ex plus 1ex minus .2ex},
	section/afterskip = {1.9ex plus .2ex},
	section/aftertitle = {\\[-0.5\baselineskip]\mbox{}\hrulefill\par}
}
\hypersetup{colorlinks, allcolors = DarkBlue}
\setcounter{secnumdepth}{-1}
\setlist{itemsep = 0.25\baselineskip, leftmargin = \parindent}
\pagestyle{fancy}\fancyhf{}\renewcommand{\headrulewidth}{0pt}
\cfoot{\thepage\ / \begin{NoHyper}\pageref{LastPage}\end{NoHyper}}
\newcommand{\gitlab}[1]{\href{https://gitlab.com/CasperVector/#1}{#1}}
\newcommand{\github}[1]{\href{https://github.com/CasperVector/#1}{#1}}

\begin{document}
算法刷题记录

\section{Leetcode 4 寻找两个正序数组的中位数 times : 3}
二分划分数组方法,在较短数组上寻找划分点,通过二分法找到将两个数组分割为大,小两部分
特别注意在二分过程中的数组越界和上下界问题!!
此题中对于正确划分的两个充分必要条件的取反可以有两种条件,两种条件的二分结构不同
caution: 注意二分是对较小数组的指标进行的不要移动大数组的下标


\section{Leetcode 350 两个数组的交集II times : 3}
1. hash table
2. sort + 双指针

\section{Leetcode 26 删除排序数组中的重复项 time : 3}
双指针
注意数组越界

\section{Leetcode 189 旋转数组 time : 3}
1. 环状替换 将第i个数填充到第i+k个位置,利用一个指标count控制被填充的数的数量,
count==n即全部替换完成,一次循环不一定能替换整个数组,利用是否回到原点来跳出循环
并从下一个元素开始新的循环

2. 翻转数组,翻转整个数组,翻转前k个元素,翻转后n-k个元素

\section{Leetcode 120 三角形最小路径和 times : 2}
动态规划 + 滚动数组
caution : 滚动时dp[i]与dp[i],dp[i-1]有关,指标从大向小滚动
可以自顶向下,也可以自底向上

\section{Leetcode 21 合并两个有序链表 times : 2}
归并排序的合并操作

\section{Leetcode 88 合并两个有序数组 times : 2}
原位合并两个有序数组,考虑从后向前归并

\section{Leetcode 1 两数之和 times : 2}
hash table

\section{Leetcode 283 移动零 times : 2}
双指针

\section{Leetcode 66 加一 times : 2}
加法运算

\section{Leetcode 242 有效的字母异位词 times : 2}
hash table

\section{Leetcode 641 设计循环双端队列 times : 2}
用数组实现,利用head和ct两个指针记录当前队列头和大小,循环利用数组实现

\section{Leetcode 49 字母异位词分组 times : 2}
利用排序后的字符串作为key,key相同的放在一起

\section{Leetcode 42 接雨水 times : 2}
1. 从左到右扫描数组,并记录最大值,更新最大值时同时更新区间内可以接的水量
再从右到左扫描到最大值处更新右侧未记录的水量
2. 利用两个数组分别记录每一点左侧最大值和右侧最大值,然后遍历一遍数组累加
每一点能接的雨水
3. 双指针法,两个指针分别从两头开始向中间移动,同时记录左侧最大值和右侧最
大值,每次更新两个指针所指的较小的位置能储存的水量,同时移动该指针

\end{document}