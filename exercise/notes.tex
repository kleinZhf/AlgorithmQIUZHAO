\documentclass[UTF8]{ctexart}
\usepackage[svgnames]{xcolor}
\usepackage{geometry, hyperref, fancyhdr, enumitem, lastpage, changepage}
\usepackage{framed, listings}
\geometry{a4paper, hmargin = 2.53cm, vmargin = 2.53cm}
\ctexset{
	section/format = {\large\bfseries},
	section/beforeskip = {3.9ex plus 1ex minus .2ex},
	section/afterskip = {1.9ex plus .2ex},
	section/aftertitle = {\\[-0.5\baselineskip]\mbox{}\hrulefill\par}
}
\hypersetup{colorlinks, allcolors = DarkBlue}
\setcounter{secnumdepth}{-1}
\setlist{itemsep = 0.25\baselineskip, leftmargin = \parindent}
\pagestyle{fancy}\fancyhf{}\renewcommand{\headrulewidth}{0pt}
\cfoot{\thepage\ / \begin{NoHyper}\pageref{LastPage}\end{NoHyper}}
\newcommand{\gitlab}[1]{\href{https://gitlab.com/CasperVector/#1}{#1}}
\newcommand{\github}[1]{\href{https://github.com/CasperVector/#1}{#1}}

\begin{document}
算法刷题记录

\section{Leetcode 4 寻找两个正序数组的中位数 times : 4}
二分划分数组方法,在较短数组上寻找划分点,通过二分法找到将两个数组分割为大,小两部分
特别注意在二分过程中的数组越界和上下界问题!!

此题中对于正确划分的两个充分必要条件的取反可以有两种条件,两种条件的二分结构不同

caution: 注意二分是对较小数组的指标进行的不要移动大数组的下标

caution:: 二分从(0, m)开始,mid的计算考虑条件判断时的数组越界问题!!!


\section{Leetcode 350 两个数组的交集II times : 4}
1. hash table

2. sort + 双指针

\section{Leetcode 26 删除排序数组中的重复项 time : 4}
双指针
注意数组越界

\section{Leetcode 189 旋转数组 time : 4}
1. 环状替换 将第i个数填充到第i+k个位置,利用一个指标count控制被填充的数的数量,
count==n即全部替换完成,一次循环不一定能替换整个数组,利用是否回到原点来跳出循环
并从下一个元素开始新的循环

2. 翻转数组,翻转整个数组,翻转前k个元素,翻转后n-k个元素

caution: 首先k对n取余!!

\section{Leetcode 120 三角形最小路径和 times : 4}
动态规划 + 滚动数组

caution : 滚动时dp[i]与dp[i],dp[i-1]有关,指标从大向小滚动
可以自顶向下,也可以自底向上

注意遍历数组的方向,++i还是--i

\section{Leetcode 21 合并两个有序链表 times : 4}
归并排序的合并操作

\section{Leetcode 88 合并两个有序数组 times : 4}
原位合并两个有序数组,考虑从后向前归并

\section{Leetcode 1 两数之和 times : 4}
hash table

\section{Leetcode 283 移动零 times : 4}
双指针

判断非零直接用逻辑判断if (target) 不要用if (target != 0)

\section{Leetcode 66 加一 times : 3}
加法运算

\section{Leetcode 242 有效的字母异位词 times : 4}
hash table

\section{Leetcode 641 设计循环双端队列 times : 4}
用数组实现,利用head和ct两个指针记录当前队列头和大小,循环利用数组实现

\section{Leetcode 49 字母异位词分组 times : 4}
利用排序后的字符串作为key,key相同的放在一起

\section{Leetcode 42 接雨水 times : 4}
1. 从左到右扫描数组,并记录最大值,更新最大值时同时更新区间内可以接的水量
再从右到左扫描到最大值处更新右侧未记录的水量

2. 利用两个数组分别记录每一点左侧最大值和右侧最大值,然后遍历一遍数组累加
每一点能接的雨水

3. 双指针法,两个指针分别从两头开始向中间移动,同时记录左侧最大值和右侧最
大值,每次更新两个指针所指的较小的位置能储存的水量,同时移动该指针

\section{Leetcode 96 不同的二叉搜索树 times : 4}
遍历根节点,每个根节点对应两个子树,相乘即为该根节点的不同的BST

\section{Leetcode 11 盛水最多的容器 times : 4}
双指针 O(n)
\begin{framed}
	\begin{lstlisting}
		int maxArea(vector<int>& height) {
			if (height.size() <= 1) return 0;
			int left = 0, right = height.size() - 1;

			int res = 0;

			while (left < right) {
				res = max(res, (right - left) * (height[left] < height[right] ? height[left++] : height[right--]));
			}

			return res;

		}
	\end{lstlisting}
\end{framed}

\section{Leetcode 785 判断二分图 times : 4}
1. 染色法,利用DFS或者BFS将由边相连的节点染成不同色,染色过程中出现矛盾即不是二分图

2. 并查集,UnionFind,用数组记录每个节点的root,将每个节点连出去的边并入同一个root
\begin{framed}
	\begin{lstlisting}[language=C++]
		Class UnionFind {
			private:
			vector<int> _root;
			public:
			UnionFind(int n) : _root(vector<int>(n)) {
				for (int i = 0; i < n; ++i) {
					_root[i] = i;
				}
			}

			int find(int x) {
				if (_root[x] = x) {
					return x;
				}
				return _root[x] = find(_root[x]);
			}

			void Union(int x, int y) {
				_root[find(x)] = find(y);
			}

			bool isConnected(int x, int y) {
				return find(x) == find(y);
			}
		}
	\end{lstlisting}
\end{framed}

\section{Leetcode 151 翻转字符串里的单词 times : 4}
1. 利用stringstream 分词和翻转句子

2. 利用双指针去空格和分词,快指针找到非空格字符并在慢指针处复制该字符,
然后将找到的每个词翻转,最后翻转整个句子即可

caution: 寻找空格的快指针到达串尾时直接退出循环,从而去掉串尾的空格。用
一个label wordCt指示是第几个单词,wordCt != 0 时在每一个单词后添空格

\section{Leetcode 125 验证回文串 times : 4}
头尾双指针,判断是否为数字和字母,判断是否相等

C++ io加速手段
\begin{framed}
	\begin{lstlisting}[language=C++]
		static int x = []() {
			std::ios::sync_with_stdio(false);
			cin.tie(NULL);
			return 0; }();
	\end{lstlisting}
\end{framed}

\section{Leetcode 127 单词接龙 times : 4}
1. BFS 将wordList存入hash table, 对当前处理的单词变换每一个位置的每一个字母,
如果在wordList中就压入队列并从wordList删除,第一次找到endWord时即为最短序列

2. 双向BFS 用两个hash table分别存储从beginWord和endWord开始搜索的word,每次
选择其中较小的table作为beginSet,搜到在endSet中的单词即可返回step

\section{Leetcode 126 单词接龙II times : 2}
BFS 同时存储路径依赖的邻接表,构造完成后DFS输出路径

\section{Leetcode 70 爬楼梯 times : 3}
动态规划+滚动数组

\section{Leetcode 15 三数之和 times : 3}
排序转化为有序数组的二数之和,然后双指针

caution:每压入一组数都要进行去重

\section{Leetcode 206 反转链表 times : 3}
1. 递归 当前节点的下个节点指向当前节点!!!
\begin{framed}
	\begin{lstlisting}[language=C++]
		ListNode* reverseList(ListNode* head) {
			if (head == NULL || head->next == NULL) return head;
	
			ListNode *p = reverseList(head->next);
			head->next->next = head;
			head->next = NULL;
	
			return p;
		}
	\end{lstlisting}
\end{framed}

2。 迭代 记录head->next,记录反转后的头节点
\begin{framed}
	\begin{lstlisting}[language=C++]
		ListNode* reverseList(ListNode* head) {
			ListNode *new_head = NULL;
			while (head) {
				ListNode *cur = head->next;
				head->next = new_head;
				new_head = head;
				head = cur;
			}
	
			return new_head;
		}
	\end{lstlisting}
\end{framed}

\section{Leetcode 24 两两交换链表中的节点 times : 3} 
\begin{framed}
	\begin{lstlisting}
		ListNode* swapPairs(ListNode* head) {
			ListNode **pp, *a, *b;
			pp = &head;

			while ((a = *pp) && (b = a->next)) {
				a->next = b->next;
				b->next = a;
				*pp = b;
				pp = &(a->next);
			}

			return head;
		}
	\end{lstlisting}
\end{framed}

\section{Leetcode 141 环形链表 times : 2}
快慢指针,若有环,快指针会追上慢指针

caution: 处理好初始条件和边界条件

\section{Leetcode 142 环形链表II times : 2}
按照141的方法判断是否有环,若有环,则将慢指针重置为head,快慢指针相同速度前进,相遇处即为
成环点。

\section{Leetcode 25 K个一组翻转链表 times : 2}
双指针确定翻转的区间:slow -> ...  -> fast, 翻转链表直到slow = fast,此时newhead为fast
的前驱,记录初始slow即为翻转链表的tail,tail->next = fast, 再继续处理fast之后的链表
\begin{framed}
	\begin{lstlisting}[language=C++]
		ListNode* reverseKGroup(ListNode *head, int k) {
			ListNode *result = new ListNode(0);

			result->next = head;
			ListNode *pre = result;
			ListNode *slow = head, *fast = head;

			while (head) {
				int c = 0;
				while (c < k && fast) {
					fast = fast->next;
					++c;
				}

				if (c != k) break;

				ListNode *tail = slow;
				ListNode *newhead = NULL;

				while (slow != fast) {
					ListNode *tmp = slow->next;
					slow->next = newhead;
					newhead = slow;
					slow = tmp;
				}

				pre->next = newhead;
				pre = tail;
				tail->next = fast;
			}

			return result->next;
		}
	\end{lstlisting}
\end{framed}

\section{Leetcode 20 有效的括号 times : 2}
栈

\section{Leetcode 155 最小栈 times : 2}
辅助栈加数据栈两个栈实现

\section{Leetcode 84 柱状图中的最大矩形 times : 2}
使用递增的栈,每次弹出栈顶元素时计算该栈顶元素能构成的最大矩形
在数组前加零使操作不会越界,在数组后加零可在最后将栈弹空。

\section{Leetcode 239 滑动窗口最大值 times : 2}
维护一个数值递减的双端队列,从队尾添加元素,从队首读取当前最大值

\end{document}