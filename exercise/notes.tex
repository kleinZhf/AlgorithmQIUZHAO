\documentclass[UTF8]{ctexart}
\usepackage[svgnames]{xcolor}
\usepackage{geometry, hyperref, fancyhdr, enumitem, lastpage, changepage}
\usepackage{framed, listings}
\geometry{a4paper, hmargin = 2.53cm, vmargin = 2.53cm}
\ctexset{
	section/format = {\large\bfseries},
	section/beforeskip = {3.9ex plus 1ex minus .2ex},
	section/afterskip = {1.9ex plus .2ex},
	section/aftertitle = {\\[-0.5\baselineskip]\mbox{}\hrulefill\par}
}
\hypersetup{colorlinks, allcolors = DarkBlue}
\setcounter{secnumdepth}{-1}
\setlist{itemsep = 0.25\baselineskip, leftmargin = \parindent}
\pagestyle{fancy}\fancyhf{}\renewcommand{\headrulewidth}{0pt}
\cfoot{\thepage\ / \begin{NoHyper}\pageref{LastPage}\end{NoHyper}}
\newcommand{\gitlab}[1]{\href{https://gitlab.com/CasperVector/#1}{#1}}
\newcommand{\github}[1]{\href{https://github.com/CasperVector/#1}{#1}}

\begin{document}
算法刷题记录

%\chapter{基础数据结构}

\section{Leetcode 4 寻找两个正序数组的中位数 times : 4}
二分划分数组方法,在较短数组上寻找划分点,通过二分法找到将两个数组分割为大,小两部分
特别注意在二分过程中的数组越界和上下界问题!!

此题中对于正确划分的两个充分必要条件的取反可以有两种条件,两种条件的二分结构不同

caution: 注意二分是对较小数组的指标进行的不要移动大数组的下标

caution:: 二分从(0, m)开始,mid的计算考虑条件判断时的数组越界问题!!!


\section{Leetcode 350 两个数组的交集II times : 4}
1. hash table

2. sort + 双指针

\section{Leetcode 26 删除排序数组中的重复项 time : 4}
双指针
注意数组越界

\section{Leetcode 189 旋转数组 time : 4}
1. 环状替换 将第i个数填充到第i+k个位置,利用一个指标count控制被填充的数的数量,
count==n即全部替换完成,一次循环不一定能替换整个数组,利用是否回到原点来跳出循环
并从下一个元素开始新的循环

2. 翻转数组,翻转整个数组,翻转前k个元素,翻转后n-k个元素

caution: 首先k对n取余!!

\section{Leetcode 120 三角形最小路径和 times : 4}
动态规划 + 滚动数组

caution : 滚动时dp[i]与dp[i],dp[i-1]有关,指标从大向小滚动
可以自顶向下,也可以自底向上

注意遍历数组的方向,++i还是--i

\section{Leetcode 21 合并两个有序链表 times : 4}
归并排序的合并操作

\section{Leetcode 88 合并两个有序数组 times : 4}
原位合并两个有序数组,考虑从后向前归并

\section{Leetcode 1 两数之和 times : 4}
hash table

\section{Leetcode 283 移动零 times : 4}
双指针

判断非零直接用逻辑判断if (target) 不要用if (target != 0)

\section{Leetcode 66 加一 times : 3}
加法运算

\section{Leetcode 242 有效的字母异位词 times : 4}
hash table

\section{Leetcode 641 设计循环双端队列 times : 4}
用数组实现,利用head和ct两个指针记录当前队列头和大小,循环利用数组实现

\section{Leetcode 49 字母异位词分组 times : 4}
利用排序后的字符串作为key,key相同的放在一起

\section{Leetcode 42 接雨水 times : 4}
1. 从左到右扫描数组,并记录最大值,更新最大值时同时更新区间内可以接的水量
再从右到左扫描到最大值处更新右侧未记录的水量

2. 利用两个数组分别记录每一点左侧最大值和右侧最大值,然后遍历一遍数组累加
每一点能接的雨水

3. 双指针法,两个指针分别从两头开始向中间移动,同时记录左侧最大值和右侧最
大值,每次更新两个指针所指的较小的位置能储存的水量,同时移动该指针

\section{Leetcode 96 不同的二叉搜索树 times : 4}
遍历根节点,每个根节点对应两个子树,相乘即为该根节点的不同的BST

\section{Leetcode 11 盛水最多的容器 times : 4}
双指针 O(n)
\begin{framed}
	\begin{lstlisting}
		int maxArea(vector<int>& height) {
			if (height.size() <= 1) return 0;
			int left = 0, right = height.size() - 1;

			int res = 0;

			while (left < right) {
				res = max(res, (right - left) * (height[left] < height[right] ? height[left++] : height[right--]));
			}

			return res;

		}
	\end{lstlisting}
\end{framed}

\section{Leetcode 785 判断二分图 times : 4}
1. 染色法,利用DFS或者BFS将由边相连的节点染成不同色,染色过程中出现矛盾即不是二分图

2. 并查集,UnionFind,用数组记录每个节点的root,将每个节点连出去的边并入同一个root
\begin{framed}
	\begin{lstlisting}[language=C++]
		Class UnionFind {
			private:
			vector<int> _root;
			public:
			UnionFind(int n) : _root(vector<int>(n)) {
				for (int i = 0; i < n; ++i) {
					_root[i] = i;
				}
			}

			int find(int x) {
				if (_root[x] = x) {
					return x;
				}
				return _root[x] = find(_root[x]);
			}

			void Union(int x, int y) {
				_root[find(x)] = find(y);
			}

			bool isConnected(int x, int y) {
				return find(x) == find(y);
			}
		}
	\end{lstlisting}
\end{framed}

\section{Leetcode 151 翻转字符串里的单词 times : 4}
1. 利用stringstream 分词和翻转句子

2. 利用双指针去空格和分词,快指针找到非空格字符并在慢指针处复制该字符,
然后将找到的每个词翻转,最后翻转整个句子即可

caution: 寻找空格的快指针到达串尾时直接退出循环,从而去掉串尾的空格。用
一个label wordCt指示是第几个单词,wordCt != 0 时在每一个单词后添空格

\section{Leetcode 125 验证回文串 times : 4}
头尾双指针,判断是否为数字和字母,判断是否相等

C++ io加速手段
\begin{framed}
	\begin{lstlisting}[language=C++]
		static int x = []() {
			std::ios::sync_with_stdio(false);
			cin.tie(NULL);
			return 0; }();
	\end{lstlisting}
\end{framed}

\section{Leetcode 127 单词接龙 times : 4}
1. BFS 将wordList存入hash table, 对当前处理的单词变换每一个位置的每一个字母,
如果在wordList中就压入队列并从wordList删除,第一次找到endWord时即为最短序列

2. 双向BFS 用两个hash table分别存储从beginWord和endWord开始搜索的word,每次
选择其中较小的table作为beginSet,搜到在endSet中的单词即可返回step

\section{Leetcode 126 单词接龙II times : 4}
BFS 同时存储路径依赖的邻接表,构造完成后DFS输出路径

\section{Leetcode 70 爬楼梯 times : 4}
动态规划+滚动数组

\section{Leetcode 15 三数之和 times : 4}
排序转化为有序数组的二数之和,然后双指针


caution:每压入一组数都要进行去重
找到一组数和双指针扫完一次区间后都需要去重

\section{Leetcode 206 反转链表 times : 4}
1. 递归 当前节点的下个节点指向当前节点!!!
\begin{framed}
	\begin{lstlisting}[language=C++]
		ListNode* reverseList(ListNode* head) {
			if (head == NULL || head->next == NULL) return head;
	
			ListNode *p = reverseList(head->next);
			head->next->next = head;
			head->next = NULL;
	
			return p;
		}
	\end{lstlisting}
\end{framed}

2。 迭代 记录head->next,记录反转后的头节点
\begin{framed}
	\begin{lstlisting}[language=C++]
		ListNode* reverseList(ListNode* head) {
			ListNode *new_head = NULL;
			while (head) {
				ListNode *cur = head->next;
				head->next = new_head;
				new_head = head;
				head = cur;
			}
	
			return new_head;
		}
	\end{lstlisting}
\end{framed}

\section{Leetcode 24 两两交换链表中的节点 times : 4} 
\begin{framed}
	\begin{lstlisting}
		ListNode* swapPairs(ListNode* head) {
			ListNode **pp, *a, *b;
			pp = &head;

			while ((a = *pp) && (b = a->next)) {
				a->next = b->next;
				b->next = a;
				*pp = b;
				pp = &(a->next);
			}

			return head;
		}
	\end{lstlisting}
\end{framed}

\section{Leetcode 141 环形链表 times : 4}
快慢指针,若有环,快指针会追上慢指针

caution: 处理好初始条件和边界条件

\section{Leetcode 142 环形链表II times : 4}
按照141的方法判断是否有环,若有环,则将慢指针重置为head,快慢指针相同速度前进,相遇处即为
成环点。

\section{Leetcode 25 K个一组翻转链表 times : 4}
双指针确定翻转的区间:slow -> ...  -> fast, 翻转链表直到slow = fast,此时newhead为fast
的前驱,记录初始slow即为翻转链表的tail,tail->next = fast, 再继续处理fast之后的链表

caution: 注意跳出循环条件,寻找fast位置时ct到了什么位置
\begin{framed}
	\begin{lstlisting}[language=C++]
		ListNode* reverseKGroup(ListNode *head, int k) {
			ListNode *result = new ListNode(0);

			result->next = head;
			ListNode *pre = result;
			ListNode *slow = head, *fast = head;

			while (head) {
				int c = 0;
				while (c < k && fast) {
					fast = fast->next;
					++c;
				}

				if (c != k) break;

				ListNode *tail = slow;
				ListNode *newhead = NULL;

				while (slow != fast) {
					ListNode *tmp = slow->next;
					slow->next = newhead;
					newhead = slow;
					slow = tmp;
				}

				pre->next = newhead;
				pre = tail;
				tail->next = fast;
			}

			return result->next;
		}
	\end{lstlisting}
\end{framed}

\section{Leetcode 20 有效的括号 times : 4}
栈

caution : 取栈顶时判断栈是否为空,最后返回栈是否为空

\section{Leetcode 155 最小栈 times : 4}
辅助栈加数据栈两个栈实现

\section{Leetcode 84 柱状图中的最大矩形 times : 4}
使用递增的栈,每次弹出栈顶元素时计算该栈顶元素能构成的最大矩形
在数组前加零使操作不会越界,在数组后加零可在最后将栈弹空。

caution : 每次弹出栈顶元素时计算一次面积$(i - S.top() - 1) * heights[tmp]$

\section{Leetcode 239 滑动窗口最大值 times : 4}
维护一个数值递减的双端队列,从队尾添加元素,从队首读取当前最大值

维护一个大顶堆

%\chapter{DFS和BFS}

\section{Leetcode 94 二叉树的中序遍历 times : 4}
1.递归DFS

2.栈,将当前节点和其左子树递归压入栈,处理栈顶元素,递归处理栈顶的右子树

\begin{framed}
	\begin{lstlisting}[language=C++]
		vector<int> inorderTraversal(TreeNode* root) {
			stack<TreeNode> S;
			vector<int> res;

			while (root || !S.empty()) {
				while (root) {
					S.push(root);
					root = root->left;
				}

				root = S.top();
				S.pop();
				res.push_back(root->val);
				root = root->right;
			}

			return res;
		}
	\end{lstlisting}
\end{framed}

\section{Leetcode 144 二叉树的前序遍历 times : 4}
1.递归DFS

2.栈,处理栈顶元素,依次压入右子树,左子树

\begin{framed}
	\begin{lstlisting}[language=C++]
		vector<int> preorderTraversal(TreeNode* root) {
			if (!root) return {};
			stack<TreeNode*> S;
			vector<int> res;
			S.push(root);

			while (!S.empty()) {
				root = S.top();
				S.pop();
				res.push_back(root->val);

				if (root->right) S.push(root->right);
				if (root->left) S.push(root->left);
			}

			return res;
		}
	\end{lstlisting}
\end{framed}

\section{Leetcode 590 N叉树的后序遍历 times : 4}
1.递归DFS

2.栈,处理根,然后将所有子树压入栈,递归处理栈顶元素,最后reverse得到的序列

\begin{framed}
	\begin{lstlisting}[language=C++]
		vector<int> postorder(Node* root) {
			if (!root) return {};
			stack<Node*> S;
			vector<int> res;
			S.push(root);

			while (!S.empty()) {
				root = S.top();
				S.pop();
				res.push_back(root->val);

				for (const auto &node : root->children) {
					S.push(node);
				}
			}

			reverse(res.begin(), res.end());
			return res;
		}
	\end{lstlisting}
\end{framed}

\section{Leetcode 589 N叉树的前序遍历 times : 4}
1.递归DFS

2.栈,处理根,然后将所有子树逆序压入栈,递归处理栈顶元素

\section{Leetcode 429 N叉树的层序遍历 times : 4}
1.BFS

2.DFS,记录当前节点的层数,将其存入对应层数的数组中

\section{Leetcode 264 丑数II times : 4}
三指针动态规划,利用三个指针i2, i3, i5记录当前乘2,乘3,乘5的位置,每次选取三个数
中较小的为下一个丑数,然后更新对应的指针。

\section{Leetcode 347 前k个高频元素 times : 3}
hash map 统计元素频次,大顶堆输出前k大元素

$\mathrm{priority\_queue<Type, Container, Functional>}$
Functional 为类对象,不能用函数指针,默认使用less方法,构成大顶堆
\begin{framed}
	\begin{lstlisting}[language=C++]
		struct cmp{
			bool operator() (Type &a, Type &b) {
				return a < b;
			}
		};

		priority_queue<Type, vector<Type>, cmp> pq;
	\end{lstlisting}
\end{framed}

\section{Leetcode 236 二叉树最近公共祖先 times : 3}
1. 分别dfs记录从root到p,q的路径,依次对比路径上的元素找到最近公共祖先

2. 递归
最近公共祖先的左右子树(或其自己)分别包含p,q节点,其他公共祖先则是只有一棵子树同时包含p,q,另一棵
子树不包含p,q

构造递归函数满足如下条件
\begin{itemize}
	\item p, q都存在在当前树中,返回公共祖先
	\item p, q只有一个存在在当前树中,返回存在的那个
	\item p, q都不存在在当前树中,返回NULL
\end{itemize}

\begin{framed}
	\begin{lstlisting}[language=C++]
		TreeNode* lowestCommonAncestor(TreeNode *root, TreeNode *p, TreeNode *q) {
			if (root == NULL || root == p || root == q) {
				return root;
			}

			TreeNode *left = lowestCommonAncestor(root->left, p, q);
			TreeNode *right = lowestCommonAncestor(root->right, p, q);

			if (left == NULL) {
				return right;
			}
			if (right == NULL) {
				return left;
			}
			if (left && right) {
				return root;
			}

			return NULL;
		}
	\end{lstlisting}
\end{framed}

\section{Leetcode 105 从前序与中序遍历序列构造二叉树 times : 3}
1. 递归
preorder root ... root->left ... root->right
inorder  root->left ... root ... root->right
preorder第一个元素定义为根,利用inorder中左右子树大小可以分离出preorder中左右子树序列,
递归构造左子树和右子树

利用hash table可以在每次递归时直接找到根节点的下标

2. 迭代
对于preorder中相邻的两个节点u,v
\begin{itemize}
	\item v 是 u的左儿子 inorder中v 在 u之前
	\item v 是前某一个节点的右儿子 
\end{itemize}
利用一个栈和一个指针进行操作,栈初始存储根节点(preorder[0]),指针初始指向inorder[0]
对preorder中每一个元素
\begin{itemize}
	\item 栈顶元素和指针元素不等,则将当前元素作为栈顶元素左儿子并入栈
	\item 栈顶元素和指针元素相等,则循环弹出栈顶元素并移动指针直至二者不等,将当前元素作为
			栈顶元素右儿子并入栈
\end{itemize}
\begin{framed}
	\begin{lstlisting}[language=C++]
		TreeNode* buildTree(vector<int>& preorder, vector<int>& inorder) {
			if (!preorder.size()) return nullptr;

			stack<TreeNode*> S;
			TreeNode *root = new TreeNode(preorder[0]);
			S.push(root);
			int index = 0;

			for (int i = 1; i < preorder.size(); ++i) {
				TreeNode *tmp = new TreeNode(preorder[i]);
				TreeNode *node = S.top();

				if (node->val != inorder[index]) {
					node->left = tmp;
				}
				else {
					while (!S.empty() && S.top()->val == inorder[index]) {
						node = S.top();
						S.pop();
						++index;
					}
					node->right = tmp;
				}

				S.push(tmp);
			}

			return root;
		}
	\end{lstlisting}
\end{framed}

%\chapter{分治递归, 二分搜索}

\section{Leetcode 77 组合 times : 3}
1. 递归 保持递增序列即可保证无重复

2. 迭代 按字典序输出每个结果,字典序写法如下,利用一个指针处理当前位置
\begin{itemize}
	\item 字典序最小序列 12...k,初始序列00...0,指针指向0
	\item 指针元素递增,如果元素大于n则指针左移
	\item 如果指针指向末尾则输出一个结果
	\item 否则指针右移且复制前一个元素
\end{itemize}

\begin{framed}
	\begin{lstlisting}[language=C++]
		vector<vector<int>> combine(int n, int k) {
			vector<vector<int>> res;
			vector<int> b(k, 0);
			int i = 0;

			while (i >= 0) {
				++b[i];
				if (b[i] > n) {
					--i;
				}
				else if (i == k - 1) {
					res.push_back(b);
				}
				else {
					++i;
					b[i] = b[i - 1];
				}
			}

			return res;
		}
	\end{lstlisting}
\end{framed}

\section{Leetcode 46 全排列 times : 3}
递归回溯
1. 利用一个数组存储每一个位置是否被使用过

2. 将nums分为左右两部分,左侧为已经确定下来的位置,右侧为子问题递归求解
\begin{framed}
	\begin{lstlisting}[language=C++]
		vector<vector<int>> res;
		vector<vector<int>> permute(vector<int>& nums) {
			dfs(0, nums);
			return res;
		}

		void dfs(int start, vector<int>& nums) {
			if (start == nums.size()) {
				res.push_back(nums);
				return;
			}

			for (int i = start; i < nums.size(); ++i) {
				swap(nums[start], nums[i]);
				dfs(start + 1, nums);
				swap(nums[start], nums[i]);
			}
		}
	\end{lstlisting}
\end{framed}

caution: dfs函数可以考虑传数组的复制,不传地址,可以实现字典序输出

\section{Leetcode 47 全排列 II times : 3}
递归回溯
将nums分为左右两部分,左侧为已经确定下来的位置,右侧为子问题递归求解
通过将nums排序来去除重复元素,每次递归process过程判断当前位置元素与代替换元素
是否相等,相等则不用替换
\begin{framed}
	\begin{lstlisting}[language=C++]
		vector<vector<int>> res;
		vector<vector<int>> premuteUnique(vector<int>& nums) {
			sort(nums.begin(), nums.end());
			dfs(0, nums);
			return res;
		}

		void dfs(int start, vector<int> nums) {
			if (start == nums.size()) {
				res.push_back(nums);
				return;
			}

			for (int i = strat; i < nums.size(); ++i) {
				if (i != start && nums[i] == nums[start]) continue;
				swap(nums[i], nums[start]);
				dfs(start + 1, nums);
			}
		}
	\end{lstlisting}
\end{framed}

\section{Leetcode 50 Pow(x, n) times : 3}
1. 二分法   递归

$Pow(x, n) = Pow(x, n/2) * Pow(x, n/2) * (n \% 2 ? x : 1)$

2. 快速幂   迭代

$n = \sum_i x_i * 2^i, \; x_i = 0, 1$

故$Pow(x, n) = \Pi_i Pow(x, x_i * 2^i)$
可以从小到大计算$Pow(x, 2^i)$,并将n不断右移一位,若$n \& 1 == 1$则将当前x乘入结果

\section{Leetcode 78 子集 times : 3}
1. 二分递归

2. 本质是列出 1 << n 中的所有数,遍历 1 << n 将每个数对应的子集存储到结果中即可

\section{Leetcode 17 电话号码的字母组合 times : 3}
分治 递归

\section{Leetcode 51 n皇后 times : 4}
逐层分治递归

利用bitmask进行压缩每一层可能的状态,依次判断当前状态是否与之前所有层的状态矛盾

\section{Leetcode 169 多数元素 times : 3}
1. 排序 O(nlogn)
求数量大于n/2的元素,则排序后n/2处元素必为结果

2. 分治 O(nlogn)
分别求左右半数元素的多数元素,然后在整个数组中验证多数元素

3. 投票算法 O(n)
维护一个多数元素的候选,遍历整个数组,
\begin{itemize}
	\item 若当前票数为0则当前元素成为候选
	\item 与候选相同的元素使投票+1
	\item 与候选不同的元素使投票-1
\end{itemize}

\section{Leetcode 860 柠檬水找零 times : 3}
贪心法找零钱

\section{Leetcode 122 买卖股票的最佳时机II times : 3}
贪心法

通过每天相对前一天的价格来决定是否买卖
\begin{itemize}
	\item 若nums[i] > nums[i - 1] 则计划在i天卖出,收益计入结果
	\item 反之在i天买入
\end{itemize}

\section{Leetcode 455 分发饼干 times : 3}
贪心法

用最小的饼干满足最小的孩子,注意每个孩子能得到的饼干数

\section{Leetcode 874 模拟行走机器人 times : 3}
利用hash table 存储障碍的坐标,根据command更新方向或者坐标,每走一步
判断当前是否为障碍,每处理一次command更新最大距离

使用$unordered\_set$处理复杂数据结构时需要构造hash函数
\begin{framed}
	\begin{lstlisting}[language=C++]
		struct myHash {
			size_t operator()(vector<int> _vec) const {
				return static_cast<size_t>(_vec[0] * 10 + _vec[1]);
			}
		};

		unordered_set<vector<int>, myHash> obList;
	\end{lstlisting}
\end{framed}

\section{Leetcode 200 岛屿数量 times : 3}
DFS

\section{Leetcode 529 扫雷游戏 times : 3}
DFS BFS

处理一个E点时,先确定周围有多少雷,有雷则更新,没雷就DFS

\section{Leetcode 55 跳跃游戏 times : 3}
1. 贪心法记录当前能到达的最大位置,若超过数组大小则返回true

2. 从后向前贪心,当前点能达到goal时将goal更新为当前点,若goal能取到0则返回true

\section{Leetcode 33 搜索旋转排序数组 times : 3}
1. 利用二分查找找到旋转点,然后在排序数组上二分查找(真实坐标为$(cur + rot) \% n$)

2. 直接二分查找
\begin{itemize}
	\item $left < mid and left <= target < mid$ 左区间有序且目标在左区间
	\item $mid < right and (mid > target or right < target)$ 右区间有序且目标不在右区间
	\item 上两种情况在左区间继续查找,否则在右区间继续查找
\end{itemize}

\section{Leetcode 74 搜索二维矩阵 times : 3}
1. 二分找到所在行,再在行内二分搜索

2. 把整个矩阵当作一维数组二分

\subsection{二分技巧}
\begin{itemize}
	\item 依据是否要判断$left == right$的情况决定循环终止条件中有没有等号
	\item 依据所需结果是否包含在某一条件中决定某一侧的二分是否包含$mid$
	\item 如果某一侧二分需要包含$mid$则通过控制计算$mid$时是否加上$frac{1}{2}$控制$left, right$相邻
		时$mid$取哪一个值,从而避免出现死循环
\end{itemize}

\section{Leetcode 153 寻找旋转排序数组最小值 times : 3}
见Leetcode 33 二分找到旋转点

\section{Leetcode 45 跳跃游戏 II times : 3}
1. 动态规划

计算某一步步数时穷举之前的格子,记录到达改点的最小步数

时间复杂度$O(n^2)$,会超时

2. 动态记录当前步数下能达到的最大点$maxlen$和多跳一步能达到的最大点$next\_max$,当前点小于
$maxlen$时增加一步,更新$maxlen$为$next\_max$。

%\chapter{动态规划}

\section{Leetcode 198 打家劫舍 times : 3}
动态规划

dp数组定义为第i个房子前能偷到的最大金额,分解为两个子问题
\begin{itemize}
	\item 偷第i间房子最大偷到$dp[i - 2] + nums[i]$
	\item 不偷第i间房子最大偷到$dp[i - 1]$
\end{itemize}
从而dp方程为$dp[i] = max(dp[i - 1], dp[i - 2] + nums[i])$

进一步节省空间,dp更新只需要前两个数,从而用三个数滚动更新dp即可

\section{Leetcode 213 打家劫舍II times : 3}
相较Leetcode 198房子改为环形排列,只需分别计算偷前n-1家和后n-1家的结果比较最大值即可

\section{股票问题合集}

\subsection{Leetcode 121 买卖股票的最佳时机 times : 3}
记录最低价格,每天更新当前最低价买入能获得的利润,然后更新最低价格

\subsection{Leetcode 122 买卖股票的最佳时机II times : 3}
1. 比较每天价格相较前一天的价格,更高则更新利润

2. 动态规划

记录每一天持有股票的最高收益$dp0$和卖出股票的最高收益$dp1$,每天滚动更新这两个值

\subsection{Leetcode 123 买卖股票的最佳时机III times : 3}
动态规划

三维dp数组$dp[i][j][k]$表示第i天卖出j次且当前手里股票状态为k时所能获得的最大利润
\begin{itemize}
	\item $dp[i][j][0] = max(dp[i - 1][j][0], dp[i - 1][j - 1][1] + prices[i])$
	\item $dp[i][j][1] = max(dp[i - 1][j][1], dp[i - 1][j][0] - prices[i])$
	\item 边界条件$dp[0][j][0] = 0$, $dp[0][j][1] = -prices[0]$
\end{itemize}

\subsection{Leetcode 309 最佳买卖股票时机含冷冻期 times : 3}
同Leetcode 122, 动态规划,不同的是买入股票$dp1$需要根据冷冻期之前的数据,所以增加
记录冷冻期之前的最大收益。

\subsection{Leetcode 188 买卖股票的最佳时机IV times : 3}
同Leetcode 123,注意当$k > \frac{n}{2}$时相当于没有买卖次数设置,可以直接返回Leetcode 122
的结果

\subsection{Leetcode 714 买卖股票的最佳时机含手续费 times : 3}
同Leetcode 122,在每次买入时计入手续费即可,注意初始化$dp1$为INTMIN,后续操作不能越界

\section{Leetcode 279 完全平方数 times : 3}
1. BFS

每一层BFS将当前元素减去某个完全平方数的结果压入队列,利用一个数组记录被访问过的数,
当访问到0时返回当前BFS层数

2. 动态规划

自底向上,每个数遍历比其小的完全平方数,更新最少数目 $O(n\sqrt{n})$

\section{Leetcode 72 编辑距离 times : 3}
动态规划

dp数组$dp[i][j]$表示word1前i个字符到word2前j个字符的编辑距离,dp方程
\begin{itemize}
	\item $dp[i][j] = min(dp[i - 1][j - 1] + word1[i] == word2[j], dp[i][j - 1], dp[i - 1][j])$
\end{itemize}

\section{Leetcode 62 不同路径 times : 3}
动态规划

\section{Leetcode 63 不同路径II times : 3}
动态规划

\section{Leetcode 980 不同路径III times : 3}
递归DFS

1. 利用一个hash tabel 存储未走过的点,DFS所有可能的路径,遇到终点且hash table 为空时即找到一条路径

2. 记录未走过空格的总数量,当遇到终点且空格数量为零时计数加一

\section{Leetcode 322 零钱兑换 times : 3}
动态规划 + 贪心法

\section{Leetcode 518 零钱兑换II times : 3}
动态规划

1. 二维dp数组$dp[i][j]$表示用前i个硬币凑出j的方法数,dp方程为
\begin{framed}
	\begin{lstlisting}[language=C++]
		if (j < coins[i-1]) {
			dp[i][j] = dp[i - 1][j];
		}
		else {
			dp[i][j] = dp[i - 1][j] + dp[i][j - coins[i - 1]];
		}
	\end{lstlisting}
\end{framed}
注意更新时只用到了当前列的上一个元素和当前行之前的元素,可以用一维滚动数组进一步优化

2. dp方程直用到了i-1中当前未更新值和i中已更新值,故可以用滚动数组优化
\begin{framed}
	\begin{lstlisting}[language=C++]
		for (auto coin : coins) {
			for (int i = 1; i <= amount; ++i) {
				if (i >= coin) dp[i] += dp[i - coin];
			}
		}
	\end{lstlisting}
\end{framed}

\section{Leetcode 191 位-1-的个数 times : 3}
位运算 $x = x\&(x - 1)$ 可以将x最低位的1置零,重复将x最低位1置零直到
x变为0


\section{Leetcode 231 2-的幂 times : 3}
2的幂有且仅有一个1,故判断清楚最低位1后是否为零
注意整形越界

\section{Leetcode 190 颠倒二进制位 times : 3}
分别对输入和输出右移和左移32次即可

\section{Leetcode 547 朋友圈 times : 3}
并查集测试

\section{Leetcode 212 单词搜索II times : 3}
利用Trie进行搜索

\section{Leetcode 410 分割数组的最大值 times : 3}
最大值最小值问题,二分查找

先确定结果的区间,然后在区间内二分查找,对每一个mid值判断是否能够完成
条件,最小或最大满足条件的值即为结果

\section{Leetcode 234 回文链表 times : 2}
快慢指针确定链表的中点,翻转后半部分链表,逐个比较前半部分和后半部分

\section{Leetcode 363 矩形区域不超过k的最大数值和 times : 2}
记录每行的前缀和,遍历lcol和rcol,计算lcol和rcol之间每行所能得到的最大数值

\section{Leetcode 1122 数组的相对排序 times : 2}
计数排序

\end{document}